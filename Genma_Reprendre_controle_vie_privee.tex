\documentclass{beamer}
\mode<presentation> {
\usepackage{color}
\definecolor{bottomcolour}{rgb}{0.21,0.11,0.21}
\definecolor{middlecolour}{rgb}{0.21,0.11,0.21}
\setbeamercolor{structure}{fg=white}
\setbeamertemplate{frametitle}[default]%[center]
\setbeamercolor{normal text}{bg=black, fg=white}
\setbeamertemplate{background canvas}[vertical shading]
[bottom=bottomcolour, middle=middlecolour, top=black]
\setbeamertemplate{items}[circle]
\setbeamertemplate{navigation symbols}{} %no nav symbols
\setbeamercolor{block title}{use=structure,fg=white,bg=structure.fg!50!red!50!blue!100!green}
\setbeamercolor{block body}{parent=normal text,use=block title,bg=block title.bg!5!white!10!bg,fg=white}
\setbeamertemplate{navigation symbols}{}
}
\usepackage{graphicx} 
\usepackage{booktabs} 
\usepackage[utf8]{inputenc}  
\usepackage[T1]{fontenc}  
\usepackage{geometry}     
%\usepackage[francais]{babel} 
\usepackage{eurosym}
\usepackage{verbatim}
\usepackage{ragged2e}
\justifying

\input{cc_beamer}

\title[Reprenons le contôle de notre vie privée sur Internet]{Reprenons le contôle de notre vie privée \\sur Internet} 
\author{Genma}

\begin{document}

%% Titlepage
\begin{frame}
	\titlepage
	\vfill
	\begin{center}
		\CcGroupByNcSa{0.83}{0.95ex}\\[2.5ex]
		{\tiny\CcNote{\CcLongnameByNcSa}}
		\vspace*{-2.5ex}
	\end{center}
\end{frame}

\begin{frame}
\frametitle{\includegraphics[scale=0.4]{./images/Genma.jpg} \ \ \  A propos de moi  }
\begin{columns}[c] 
\column{.55\textwidth} 
\textbf{Où me trouver sur Internet?}
\begin{itemize}
\item Le Blog de Genma : http://genma.free.fr
\item Twitter : http://twitter.com/genma
\end{itemize}
\textbf{Mes projets-contributions}
\\ Plein de choses dont:
\begin{itemize}
\item Des conférences sur plein de thèmes différents
%\item A.I.\up{2} Apprenons l'Informatique, Apprenons Internet
\end{itemize}
\column{.5\textwidth} 
\includegraphics[width=5cm,height=5cm]{./images/blog.png} 
\end{columns}
\end{frame}

%----------------------------------------------------------------------------------------
\begin{frame}
\begin{center}
\Huge{Internet, c'est quoi?}
\end{center}
\end{frame}

%----------------------------------------------------------------------------------------
\begin{frame}
\frametitle{Internet, un réseau de réseau}
\begin{itemize}
\justifying{
\item Internet c'est un réseau de réseau d'ordinateurs connectés entre eux.
\item Il y a d'un côté les serveurs, des gros ordinateurs, sur lesquels il y a des sites Internet.
\item Et de l'autre, il y a "nous", avec notre PC, notre tablette, notre smartphone...
}
\end{itemize}
\end{frame}

\begin{frame}
\Huge{\centerline{Toutes ces traces qu'on laisse}}
\Huge{\centerline{sur Internet... sans le savoir}}
\end{frame}

%----------------------------------------------------------------------------------------
\begin{frame}
\frametitle{Les traces de navigation \emph{locales}}

\begin{block}{Quand on va sur \emph{Internet}}
\justifying{
Plein de fichiers sont créés : 
\begin{itemize}
\item Historique des pages visitées, 
\item Données saisies dans les formulaires et barres de recherche,
\item Les mots de passe conservés, 
\item La liste des téléchargements, 
\item Les cookies, 
\item Les fichiers temporaires...)
\end{itemize}
}
\end{block}
\justifying{Tout ce que l'on fait depuis son navigateur, est, par défaut, conservé sur notre ordinateur, tablette, smartphone...}
\end{frame}


%----------------------------------------------------------------------------------------
\begin{frame}
\frametitle{Les logs de connexions}

\begin{block}{Les traces laissé sur les sites Internets}
\justifying{
Les serveurs Internet gardent différentes traces dont : 
\begin{itemize}
\item l'Adresse IP
\item les heures et dates de connexions
\item les informations saisies...
\item le naivgateur, son modèle, le système d'exploitation...
\end{itemize}
}
\end{block}
\end{frame}
%----------------------------------------------------------------------------------------
\begin{frame}
\frametitle{Les traces écrites}

\begin{block}{Sur les réseaux sociaux, les blogs, les forums...}
\justifying{
Sur tous ces comptes que l'on en a en ligne : 
\begin{itemize}
\item On commente, on réagit ;
\item On "like" ;
\item On ajoute des photos, des vidéos.
\end{itemize}
}
\end{block}
\justifying{Ce sont autant de traces que l'on peut lier à nous.}
\end{frame}

%----------------------------------------------------------------------------------------
\begin{frame}
\frametitle{L'image que je donne de moi}

\justifying{
\begin{block}{\emph{Googler} "son nom"}
\begin{itemize}
\item Les résultats apparaissant sont-ils bien ce que l'on souhaite?
\end{itemize}
\end{block}
}
\begin{center}
\includegraphics[scale=0.3] {./images/Google01.png}
\\
\includegraphics[scale=0.3] {./images/Google02.png}
\end{center}
\end{frame}

%----------------------------------------------------------------------------------------
\begin{frame}
\frametitle{Adage}
\begin{block}{Les paroles s'envolent, les écrits restent}
\begin{itemize}
\justifying{
\item Cet adage est encore plus vrai avec Internet.
\item Il faut partir du principe que ce que l'on dit sera toujours accessible, même des années après.
\item Tout ce qui est sur Internet est public ou le sera (même si c'est "privé". Les conditions d'utilisation évoluent. cf. Facebook).
}
\end{itemize}
\justifying{Rq : Il ne faut donc pas abuser de la liberté d'expression et rester respectueux des lois en vigueurs.}
\end{block}
\end{frame}

%----------------------------------------------------------------------------------------
\begin{frame}
\frametitle{Les mails - courriers électroniques}

\begin{block}{Un mail que l'on envoit, c'est une carte postale}
\justifying{ "On" sait
\begin{itemize}
\item qui écrit à qui ;
\item quand ;
\item pour se dire quoi.
\end{itemize}
}
\justifying{Ex : Gmail lit le contenu des mails pour afficher de la publicité ciblée.}
\end{block}

\begin{block}{Le facteur peut lire la carte postale}
\justifying{ "On" peut
\begin{itemize}
\item envoyer un mail au nom de quelqu'un d'autres ;
\item lire les mails qui circulent sur un réseau...
\end{itemize}
}
\end{block}

\end{frame}
%----------------------------------------------------------------------------------------
\begin{frame}
\frametitle{Les données qui transitent \emph{en clair} sur le Web}
\begin{block}{Quand on consulte un site Internet}
\justifying{
 Le site Internet sait :
\begin{itemize}
\item D'où l'on vient (pays, adresse exacte)
\item La langue que l'on parle, l'heure de l'ordinateur, son modèle...
\end{itemize}
}
\end{block}
\begin{block}{Les connexion http}
\justifying{
Sans le "cadenas" dans la barre d'adresse :
\begin{itemize}
\item Le mot de passe circule "en clair"
\item Avec \emph{un logiciel}, \emph{un pirate} peut récupérer le mot de passe.
\end{itemize}
}
\end{block}
\end{frame}
%----------------------------------------------------------------------------------------
\begin{frame}
\frametitle{Cloud - l'informatique dans les nuages}
\begin{block}{Définition du cloud}
\justifying{
\begin{itemize}
\item Le \emph{Cloud} , c'est l'ordinateur d'un autre.
\end{itemize}
}
\begin{center}
\includegraphics[scale=0.25] {./images/cloud.png} 
\end{center}
\end{block}

\begin{block}{Les \emph{problèmes} du cloud}
\justifying{
Le stockage est gratuit
\begin{itemize}
\item les documents sont analysés (pour de la publicité, de l'espionnage industriel... etc.).
\item Nos données peuvent être piratées et diffusées dans la nature?
\item Si le service ferme, que deviennent nos données?
\end{itemize}
}
\end{block}

\end{frame}
%----------------------------------------------------------------------------------------
\begin{frame}
\begin{center}
\Huge{Les métadonnées}\\~\\
\LARGE{Ces données cachées des documents}
\end{center}
\end{frame}

\begin{frame}
\frametitle{Les métadonnées}
\begin{block}{Qu'est-ce qu'une metadonnée ?}
\justifying{
Une métadonnée est une information qui caractérise une donnée. 
\\~\\
Prenons un exemple : lorsque vous créez un PDF, en général, des données additionnelles sont ajoutées à votre fichier : le nom du logiciel producteur, votre nom, la date de production, la description de votre document, le titre de votre document, la dernière date de modification, … ce sont des métadonnées. 
\\~\\
Vous n'avez peut-être pas envie de partager ces informations lorsque vous partagez votre fichier.}
\end{block}
\end{frame}

%------------------------------------------------
\begin{frame}
\frametitle{Metadata Photo}
\begin{center}
\includegraphics[scale=0.5] {./images/Metadata.png} 
\end{center}
\end{frame}

\begin{frame}
\frametitle{Metadata Photo}
\begin{center}
\includegraphics[scale=0.5] {./images/exif-metadata.jpg} 
\end{center}
\end{frame}

\begin{frame}
\frametitle{Géolocalisation des chats}
\begin{center}
\includegraphics[scale=0.3] {./images/Chat_geolocalisaion.png}
\end{center}
\end{frame}

\begin{frame}
\frametitle{Metadata Word 1/2}
\begin{center}
\includegraphics[scale=0.5] {./images/Word01.jpg} 
\end{center}
\end{frame}

\begin{frame}
\frametitle{Metadata Word 2/2}
\begin{center}
\includegraphics[scale=0.5] {./images/Word02.jpg}
\end{center}
\end{frame}

%------------------------------------------------
\begin{frame}
\frametitle{Les métadonnées}
\begin{block}{Pourquoi les métadonnées sont elles un risque pour notre vie privée?}
\justifying{
Les métadonnées dans un fichier peuvent en dire beaucoup sur vous. Les appareils photos enregistrent des données sur le moment où une photo a été prise et quel appareil photo a été utilisé. 
\\~\\
Les documents bureautiques ajoutent automatiquement l'auteur et diverses informations sur la société aux documents et feuilles de calcul. 
\\~\\
Peut-être que vous ne voulez pas divulguer ces informations sur le web?
}
\end{block}
\end{frame}

\begin{frame}
\begin{center}
\Huge{Quand on fait une recherche dans Google...}
\end{center}
\end{frame}

%----------------------------------------------------------------------------------------
\begin{frame}
\frametitle{Google}

\begin{block}{Découvrez comment Google vous voit}
\justifying{
Google tente de créer un profil de base de vous, selon votre âge, votre sexe, vos centres d’intérêt. C’est avec ces données que Google vous « sert » des annonces pertinentes. Vous pouvez examiner la façon dont Google vous voit ici  :
\url{https://www.google.com/ads/preferences/}
}
\end{block}

\begin{block}{Découvrez l’historique de votre géolocalisation}
\justifying{
Si vous utilisez Android, votre appareil mobile peut envoyer à Google des informations de géolocalisation et de vitesse de déplacement d’un point à l’autre. Vous pouvez voir l’historique complet de vos « positions » et les exporter ici  :\\
\url{https://maps.google.com/locationhistory}
}
\end{block}
\end{frame}
\begin{frame}
\frametitle{Google}
\begin{block}{Découvrez l’intégralité de votre historique de recherches Google}
\justifying{
Google enregistre jusqu’à la moindre recherche que vous faites. Par-dessus le marché, 
Google enregistre toutes les pubs Google sur lesquelles vous avez cliqué. L’historique est à votre disposition ici  :\\
\url{https://history.google.com}
}
\end{block}
\begin{block}{Découvrez tous les appareils qui ont accédé à votre compte Google}
\justifying{
Si vous craignez que quelqu’un d’autre ait pu utiliser votre compte, vous pouvez trouver la liste de tous les appareils qui ont accédé à votre compte Google, leur adresse IP et leur emplacement approximatif  :\\
\url{https://security.google.com/settings/security/activity}
 }
\end{block}
\end{frame}
\begin{frame}
\frametitle{Google}
\begin{block}{Découvrez toutes les applications et les extensions qui ont accès à vos données Google}
\justifying{
Ceci est une liste de toutes les applications qui ont tout type d’accès à vos données. Vous pouvez voir le type exact de permissions accordées à l’application et révoquer l’accès à vos données en suivant ce lien  :\\
\url{https://security.google.com/settings/security/permissions}
}
\end{block}
\end{frame}

%----------------------------------------------------------------------------------------
\begin{frame}
\begin{center}
\Huge{Sur Internet, si c'est gratuit, c'est VOUS le produit }
\end{center}
\end{frame}
%----------------------------------------------------------------------------------------
\begin{frame}
\frametitle{Comment est-on pisté?}

\justifying{
\begin{block}{Toutes les publicités nous espionnent}
\begin{itemize}
\item Le bouton Like de Facebook : il permet à FaceBook de savoir que vous avez visité ce site, même si vous n'avez pas cliqué sur ce bouton.
\item Même si vous vous êtes correctement déconnecté de Facebook.
\item De même pour le bouton le +1 de Google, les scripts de Google Analytics, 
\item Tous les publicité, Amazon...
\end{itemize}
\end{block}
}
\begin{center}
\includegraphics[scale=0.3] {./images/Facebook_like.png}
\end{center}
\end{frame}

\begin{frame}
\frametitle{Lightbeam}
\begin{center}
\includegraphics[scale=0.5] {./images/lightbeam.png}
\end{center}
\end{frame}

%========================================================================================
\begin{frame}
\Huge{\centerline{Comment se protéger ?}}
\Huge{\centerline{Un peu d'hygiène numérique}}
\end{frame}

%----------------------------------------------------------------------------------------
\begin{frame}
\frametitle{Navigateur}

Utilisez un navigateur respectueux de nos données personnelles : Firefox.

\begin{block}{Pourquoi Firefox?}
\justifying{
\begin{itemize}
\item La navigation privée permet de ne pas garder de traces sur l'ordinateur (mais ça ne suffit pas).
\item On peut ajouter des extensions anti-tracking et anti-pubs (Ghostery/Request Policy, Adblock...)
\end{itemize}
}
\end{block}
\justifying{}
\end{frame}

%----------------------------------------------------------------------------------------
\begin{frame}
\begin{center}
\Huge{Installer des extensions \\ pour Firefox }
\end{center}
\end{frame}

%----------------------------------------------------------------------------------------
\begin{frame}
\frametitle{AdBlock - block 1/2}
Page avec publicité :
\begin{center}
\includegraphics[scale=0.4] {./images/Adblock01.png}
\end{center}

\end{frame}

%----------------------------------------------------------------------------------------
\begin{frame}
\frametitle{AdBlock - Microblock 2/2}
Bloque les publicités. Allège les pages.

\begin{center}
\includegraphics[scale=0.4] {./images/Adblock02.png}
\end{center}
\end{frame}

%----------------------------------------------------------------------------------------
\begin{frame}
\frametitle{Ghostery}

Bloque tous les trackers associés au site.
\begin{center}
\includegraphics[scale=0.4] {./images/Ghostery_tracker.png}
\end{center}
\end{frame}

%----------------------------------------------------------------------------------------
\begin{frame}
\frametitle{HttpsEverywhere}

Avoir une connexion httpS dès que possibe
\begin{center}
\includegraphics[scale=0.5] {./images/httpseverywhere.jpg}
\end{center}
\end{frame}

%----------------------------------------------------------------------------------------
\begin{frame}
\frametitle{Utiliser des moteurs de recherche plus respectueux de la vie privée (DuckDuckGo...)}

\begin{block}{Les alternatives à Google}
\justifying{
\begin{itemize}
\item Duckduckgo \url{https://duckduckgo.com}
\item Qwant \url{https://www.qwant.com}
\item Framabee \url{https://framabee.org} ou TontonRoger \url{https://tontonroger.org/}
\end{itemize}
}
\end{block}
\end{frame}

%----------------------------------------------------------------------------------------
\begin{frame}
\begin{center}
\frametitle{Duckduckgo - Google tracks you. We don't.}

\url{https://duckduckgo.com/}
\\
\includegraphics[scale=0.6] {./images/DuckDuckGo.jpg}
\end{center}
\end{frame}

%----------------------------------------------------------------------------------------
\begin{frame}
\frametitle{Différents modèles de menace}
\begin{block}{Répondre aux questions}
\justifying{
\begin{itemize}
\item Quelles sont les données et informations que j'estime personnelles - confidentielles? 
\item Qu'est ce que je suis près à apprendre et à faire pour les protéger?
\end{itemize}
}
\end{block}
\justifying{}
\end{frame}

%----------------------------------------------------------------------------------------
\begin{frame}
\begin{center}
\Huge{Autres conseils }
\end{center}
\end{frame}

%----------------------------------------------------------------------------------------
\begin{frame}
\begin{center}
\Huge{Utiliser un pseudonyme }
\end{center}
\end{frame}

%----------------------------------------------------------------------------------------
\begin{frame}
\frametitle{Le pseudonymat}

\begin{block}{Défintions}
\begin{itemize}
\justifying{
\item Contraction des termes pseudonyme et anonymat, le terme de pseudonymat reflète assez bien la volonté contradictoire d’être un personnage publique et de rester anonyme...
\item Un pseudonyme, c'est aussi une identité publique, qui est associée à un ensemble cohérent de compte qui forme un tout : un blog, un compte Twitter, un compte Facebook...
}
\end{itemize}
Avoir un pseudonyme ne veut pas dire faire et dire n'importe quoi.
\\~\\Il en va de l'image que je renvoie, que je donne de moi et de ma crédibilité présente et à venir.
\end{block}
\end{frame}

%----------------------------------------------------------------------------------------
\begin{frame}
\frametitle{Les avantages du pseudonymat}

\begin{block}{Ce que permet le pseudonymat}
\justifying{Il permet de cloisonner sa vie numérique.}
\begin{itemize}
\justifying{
\item On a une une identité civile en ligne (nom prénom) avec le strict minimum.
\item Et une identité publique, un pseudonyme, qui permet d'avoir une activité plus fournie.
}
\end{itemize}
\end{block}
\justifying{
Ne pas oublier d'avoir une adresse mail qui n'est pas de la forme prénom.nom (sinon on perd l'intérêt du pseudonyme).
}
\end{frame}

%----------------------------------------------------------------------------------------
\begin{frame}
\frametitle{Plusieurs pseudonymes}

\justifying{
Quand on crée un compte sur un site, on peut envisager de saisir des informations nominatives spécifiques à ce site. On aura alors un pseudonyme par type de communauté fréquenté (jeu vidéo, informatique, de rencontres...).
\\~\\
Si il y a un problème (\emph{compte piraté}), on limitera le risque de diffusion des informations personnelles.
}
\end{frame}

%----------------------------------------------------------------------------------------
\begin{frame}
\frametitle{Pseudonymat et célébrité}

\justifying{Nombreux sont les célébrités du monde de la télévision, cinéma, musique... Et Internet?}
\begin{block}{Des pseudonymes internet \emph{connus}}
\begin{itemize}
\justifying{
\item Maitre Eolas, l'avocat
\item Zythom, l'expert judiciaire
\item Boulet, dessinateur
\item ...
}
\end{itemize}
\end{block}
\justifying{Et beaucoup d'autres, dans les communautés geek, hackers...}
\end{frame}

%----------------------------------------------------------------------------------------
\begin{frame}
\frametitle{Les limites du pseudonymat}

\begin{block}{Un pseudonymat c'est contraignant}

\justifying{On est très facilement tracés et reliés à sa véritable identité (via l'adresse IP).}
\begin{itemize}
\justifying{
\item Pour avoir un pseudonymat parfaitement cloisonné, il faut utiliser différentes techniques avancées...
}
\end{itemize}
\end{block}

\begin{block}{NE JAMAIS faire d'erreur}
\begin{itemize}
\justifying{
\item On ne dévoile pas son pseudonyme a des personnes qui connaissent notre identité civile.
\item On ne dévoile pas son visage en publique....
}
\end{itemize}
\end{block}
\justifying{
Le pseudonymat est donc on ne peut plus relatif et tout dépende de ce que l'on souhaite comme pseudonymat.
}
\end{frame}

%----------------------------------------------------------------------------------------
\begin{frame}
\begin{center}
\Huge{Une solution simple reste d'être moins \emph{bavard}}
\end{center}
\end{frame}

%----------------------------------------------------------------------------------------
\begin{frame}
\begin{center}
\Huge{D'un peu plus complexe...\\ à très technique}
\end{center}
\end{frame}

\begin{frame}
\begin{center}
\includegraphics[scale=0.4] {./images/LogoCafeViePrivee.jpg}
\end{center}

\justifying{Chaque exemple cité après, c'est entre 1h à 3h d'ateliers...}
\end{frame}

%----------------------------------------------------------------------------------------
\begin{frame}
\frametitle{Chiffrer ses disque durs (TrueCrypt...)}
\begin{center}
\includegraphics[scale=0.4] {./images/Truecrypt18.png}
\end{center}
\end{frame}

%----------------------------------------------------------------------------------------
\begin{frame}
\frametitle{Chiffrer ses disque durs (TrueCrypt...)}
\begin{center}
\includegraphics[scale=0.4] {./images/Truecrypt18.png}
\end{center}
\end{frame}

%----------------------------------------------------------------------------------------
\begin{frame}
\frametitle{Chiffrer ses mails  (GPG...)}
\begin{center}
\includegraphics[scale=0.5] {./images/chiffrement_mail.jpg}
\end{center}
\end{frame}

%----------------------------------------------------------------------------------------
\begin{frame}
\frametitle{HttpS}
\begin{center}
\includegraphics[scale=0.5] {./images/https.jpg}
\end{center}
\end{frame}

%----------------------------------------------------------------------------------------
\begin{frame}
\begin{center}
\Huge{Quelques mots sur Tor ? }
\\~\\ \includegraphics[scale=0.4]{./images/logo_tor.jpg}
\end{center}
\huge{Attention : la présentation \emph{complète} dure une bonne heure et demi...}
\end{frame}

%----------------------------------------------------------------------------------------
\begin{frame}
\frametitle{Comment fonctionne Tor ?}
\begin{center}
\includegraphics[keepaspectratio,width=\textwidth, height=.8\textheight]{images/tor-keys1}
\end{center}
\end{frame}

%----------------------------------------------------------------------------------------
\begin{frame}
\frametitle{A quoi sert TOR?}

\begin{block}{Ce que l'usage de Tor permet de faire}
\justifying{
\begin{itemize}
\justifying{
\item  d'échapper au fichage publicitaire,
\item  de publier des informations sous un pseudonyme,
\item  d'accéder à des informations en laissant moins de traces,
\item  de déjouer des dispositifs de filtrage (sur le réseau de son entreprise, de sa Université, en Chine ou en France…),
\item  de communiquer en déjouant des dispositifs de surveillances,
\item  de tester son pare-feu,
\item  … et sûrement encore d'autres choses.
}
\end{itemize}
$\Rightarrow$ Tor dispose également d'un système de « services cachés » qui permet de fournir un service en cachant l'emplacement du serveur.
}
\end{block}
\end{frame}

%----------------------------------------------------------------------------------------
\begin{frame}
\frametitle{Télécharger le Tor Browser}
\justifying{
Toutes les versions (dans différentes langues, différents OS) sont disponibles sur le site du projet : 
\\ \url{https://www.torproject.org/}
\\ Rq : Il existe la possibilité de le recevoir par mail...
}
\begin{center}
\includegraphics[scale=0.5]{./images/tor2.jpg}
\end{center}
\end{frame}

%----------------------------------------------------------------------------------------
\begin{frame}
\frametitle{Lancer le Tor Browser}
\begin{center}
\includegraphics[scale=0.3]{./images/tor_browser03.jpg}
\end{center}
\end{frame}
%----------------------------------------------------------------------------------------

\begin{frame}
\frametitle{Utiliser Tor - Tails}
\justifying{
Tails (The Amnesic Incognito Live System) est un système d'exploitation complet basé sur Linux et Debian, en live.
}
\begin{center}
\includegraphics[scale=0.3]{./images/tails.jpg}
\\~\\
\url{https://tails.boom.org}
\end{center}
\end{frame}

%----------------------------------------------------------------------------------------
\begin{frame}
\Huge{\centerline{Merci de votre attention.}}
\Huge{\centerline{Place aux questions.}}
\end{frame}
%============================================================================================
\begin{frame}
\Huge{\centerline{ANNEXES}}
\end{frame}
%----------------------------------------------------------------------------------------
\begin{frame}
\frametitle{La navigation en mode privée}

\justifying{
\begin{block}{Quelles données ne sont pas enregistrées durant la navigation privée ?}
\begin{itemize}
\item pages visitées ;
\item saisies dans les formulaires et la barre de recherche ;
\item mots de passe ; 
\item liste des téléchargements ; 
\item cookies ;
\item fichiers temporaires ou tampons.
\end{itemize}
\end{block}
}
\end{frame}


%----------------------------------------------------------------------------------------
\begin{frame}
\frametitle{Effacer ses métadonnées}

\begin{block}{Effacer de façon sécurisé}
\justifying{
\begin{itemize}
\item Pour le formatage : Shred
\item Pour les métadonnées : MAT
\end{itemize}
}
\end{block}
\justifying{}
\end{frame}

%------------------------------------------------
\begin{frame}
\frametitle{Le logiciel MAT}
\begin{block}{Le logiciel MAT}
\justifying{
MAT est une boîte à outil composé d'une interface graphique, d'une version en ligne de commande et d'une bibliothèque.
\\~\\
MAT crée automatiquement une copie des documents originaux dans une version nettoyée (laissant intact les originaux). 
\\~\\
MAT est fournit par défaut dans le live-cd Tails. 
}
\end{block}
\end{frame}

%------------------------------------------------
\begin{frame}
\frametitle{Le logiciel MAT}
\begin{center}
\includegraphics[scale=0.3] {./images/Mat.png} 
\end{center}
\end{frame}



%------------------------------------------------
\begin{frame}
\frametitle{Comment vérifier rapidement la sécurité d'un site?}
\begin{block}{La check-liste}
\begin{itemize}
\justifying{
\item Le site a-t-il une connexion en https? (SSL).
\item Y-a-t-il intégration d'éléments extérieurs au site en lui-même?
\item Le site utilise-t-il Google Analytics?
\item Le site utilise-t-il Google Fonts?
\item Le site utilise-t-il des régies publicitaires?
\item Le site utilise-t-il Cloudflare?
\item Le DNS est-il géré par Cloudflare?
\item Le site présente-t-il une politique de confidentialité?
\item Le site utilise-t-il les cookies?
\item Le site utilise-t-il des scripts javascripts?
}
\end{itemize}
\end{block}
\end{frame}


%----------------------------------------------------------------------------------------
\begin{frame}
\Huge{\centerline{L'authentification forte}}
\end{frame}

\begin{frame}
\frametitle{L'authentification forte}

\begin{block}{Différents termes, un même usage}
Double authentification, Connexion en deux étapes, 2-Step Verification
\end{block}

\begin{block}{Exemple avec Google} 
\justifying{
Google permet aux utilisateurs d'utiliser un processus de vérification en deux étapes.
\begin{itemize}
\item La première étape consiste à se connecter en utilisant le nom d'utilisateur et mot de passe. Il s'agit d'une application du facteur de connaissance.
\item Au moment de la connexion Google envoit par SMS un nouveau code unique. Ce nombre doit être entré pour compléter le processus de connexion. 
\end{itemize}
Il y a aussi une application à installer qui génère un nouveau code toutes les 30 secondes.
}
\end{block}
\end{frame}

%----------------------------------------------------------------------------------------
\begin{frame}
\frametitle{L'authentification forte}
\begin{block}{Autres services implémentant cette fonctionnalité}
\begin{itemize}
\item Web : Facebook, Twitter, Linkedin, Paypal
\item Banque : envoit d'un code par SMS
\end{itemize}
\end{block}
\end{frame}

%----------------------------------------------------------------------------------------
\begin{frame}
\Huge{\centerline{Des fuites de données personnelles}}
\end{frame}
%------------------------------------------------
\begin{frame}
\frametitle{Le sexe par exemple. C’est perso ou pas ?}

\begin{block}{Billet Tout à cacher par Kiteoa \url{http://reflets.info/}}
\begin{itemize}
\justifying{
\item En 2012, les identifiants et les mots de passe d’utilisateurs de Youporn ont été diffusés sur Pastebin...
}
\end{itemize}
\end{block}
\begin{center}
\includegraphics[scale=1]{./images/youporn.png}
\end{center}
\end{frame}

%------------------------------------------------
\begin{frame}
\frametitle{Le sexe par exemple. C’est perso ou pas ?}

\begin{block}{Billet Tout à cacher par Kiteoa \url{http://reflets.info/}}
\begin{itemize}
\justifying{
\item Une boutique en ligne de type sexshop s'est fait piratée... Chacun a le droit de garder pour lui le fait qu’il achète (ou pas) des surtout si « chacun » a utilisé son mail professionnel pour passer commande…}
\end{itemize}
\end{block}
\begin{center}
\includegraphics[scale=1]{./images/lesexshopquivamal.png}
\end{center}
\end{frame}

%============================================================================================
\end{document}
